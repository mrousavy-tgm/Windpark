%!TEX root=../protocol.tex	% Optional

\section{Lösung}

\subsection{Projekt migrieren}

Da die Aufgabe die bestehende Aufgabe \textbf{'Message Oriented Middleware'} erweitert, wird die gesamte Codebase als Grundlage weiterverwendet.

Das Projekt bestand aus einem Windpark welcher eine Liste aus Windrädern besitzt.

Außerdem gibt es eine Zentrale welche über die Middleware die Informationen des Windparks erhält.

Dies soll erweitert werden, sodass statt der \textbf{Message Oriented} Middleware eine \textbf{Document Oriented} Middleware verwendet wird.

\clearpage
\subsubsection{Neues Projekt erstellen}

Das erstellen des neuen Projektes erfolgt mit dem \textbf{Spring Initializr}.

Folgende allgemeine Projekt Einstellungen habe ich konfiguriert:

\begin{figure}
    \caption{Spring Initializr Projektkonfiguration}
    \includegraphics[width=15cm]{images/initializr-general}
    \centering
\end{figure}

Sowie folgende Frameworks für das Windpark Projekt:

\begin{figure}
    \caption{Spring Initializr Framework Konfiguration}
    \includegraphics[width=15cm]{images/initializr-frameworks}
    \centering
\end{figure}

\clearpage
\subsection{Windpark}

\subsubsection{Allgemein}

Eine Spring Web Application wird nun einen einzelnen Windpark darstellen.

Es wird also zusätzlich zu folgender Application Klasse:

\begin{code}{java}
    package com.mrousavy.wpwindpark;

    import org.springframework.boot.SpringApplication;
    import org.springframework.boot.autoconfigure.SpringBootApplication;

    @SpringBootApplication
    public class WpWindparkApplication {
        public static void main(String[] args) {
            SpringApplication.run(WpWindparkApplication.class, args);
        }
    }
\end{code}

.., welche mittels des \textit{Spring Initializr} automatisch erstellt wurde, noch ein \textbf{ReST Controller} erstellt.

\subsubsection{ReST Controller}

Dieser \textbf{ReST Controller} kümmert sich um alle Anfragen über \textit{HTTP/HTTPS}, und liefert einen \texttt{String} zurück.

Definiert werden \textbf{ReST Controller} in \textbf{Spring} mittels der \texttt{@RestController} Annotation:

\begin{code}{java}
    @RestController
    public class WindparkRestController {
\end{code}

Sobald also die Spring Application über die URL beispielsweise in einem Browser aufgerufen wird, sucht \textbf{Spring} einen passenden \textbf{ReST Controller} um die Anfrage zu verarbeiten. Dies gelangt zur Laufzeit mittels der vorhin erwähnten \texttt{@RestController} Annotation.

Wir können einerseits einen \textit{Index} (also die Startseite, beispielsweise erreichbar unter \texttt{172.0.1.2}), oder eine Subseite (beispielsweise erreichbar unter \texttt{172.0.1.2/hello}) definieren. Hierbei verwenden wir die Spring Web Annotation \texttt{@GetMapping(String)}.

Um einen Wert (\texttt{String}) zurückzugeben, wird die Spring Web Annotation {\texttt{@ResponseBody}} verwendet.

Also können wir beispielsweise folgende Funktion definieren:

\begin{code}{java}
    @GetMapping("/hello")
    @ResponseBody
    public String sayHello() {
        return "<p>Hello world!</p>";
    }
\end{code}

Der Windpark Controller ist so aufgebaut, dass damit gerechnet wird, dass ein vorgefertigtes \texttt{XML} File auf dem Server abgespeichert ist, welches über eine \textit{ReST GET-Function} abgerufen wird.

Um eine Text-Datei in Java 7 zu lesen, wird der \texttt{BufferedReader} verwendet. Es wird mit der \texttt{BufferedReader.getLine()} Methode gearbeitet.

\begin{code}{java}
    package com.mrousavy.wpwindpark;

    import org.springframework.web.bind.annotation.GetMapping;
    import org.springframework.web.bind.annotation.ResponseBody;
    import org.springframework.web.bind.annotation.RestController;

    @RestController
    public class WindparkRestController {

        private final String FILE_NAME = "parknodedata.xml";

        public String readXml() throws IOException {
            try (BufferedReader br = new BufferedReader(new FileReader(FILE_NAME))) {
                StringBuilder builder = new StringBuilder();

                String sCurrentLine;
                while ((sCurrentLine = br.readLine()) != null) {
                    builder.append(sCurrentLine);
                }

                return builder.toString();
            }
        }
    }
\end{code}

\subsubsection{JSON}

Außerdem wird die Dependency \texttt{org.JSON} verwendet, um das XML zu einem JSON Format zu konvertieren:

\begin{code}{properties}
    dependencies {
        compile('org.springframework.boot:spring-boot-starter-data-rest')
        compile('org.springframework.boot:spring-boot-starter-web')
        compile('org.jdom:jdom2:2.0.6')
        compile('xerces:xercesImpl:2.11.0')
        compile group: 'org.json', name: 'json', version: '20180130'
        testCompile('org.springframework.boot:spring-boot-starter-test')
    }
\end{code}

Der vollständige Code des \textbf{ReST Controllers} sieht nun folgendermaßen aus:

\begin{code}{java}
    @RestController
    public class WindparkRestController {

        private final String FILE_NAME = "parknodedata.xml";

        public String readXml() throws IOException {
            try (BufferedReader br = new BufferedReader(new FileReader(FILE_NAME))) {
                StringBuilder builder = new StringBuilder();

                String sCurrentLine;
                while ((sCurrentLine = br.readLine()) != null) {
                    builder.append(sCurrentLine);
                }

                return builder.toString();
            }
        }

        @GetMapping("/xml")
        @ResponseBody
        public String getXml() {
            System.out.println("Working Directory = " + System.getProperty("user.dir"));

            try {
                return readXml();
            } catch (Exception e) {
                e.printStackTrace();
                return "<error>" + e.getMessage() + "</error>";
            }
        }


        @GetMapping("/json")
        @ResponseBody
        public String getJson() {
            System.out.println("Working Directory = " + System.getProperty("user.dir"));

            try {
                String xml = readXml();

                JSONObject json = XML.toJSONObject(xml);
                String prettyJson = json.toString(4);
                System.out.println("Read JSON: " + prettyJson);
                return prettyJson;
            } catch (Exception e) {
                e.printStackTrace();
                return "<error>" + e.getMessage() + "</error>";
            }
        }
    }
\end{code}

\subsubsection{Port}

Da wir mehrere Windparks auf einem Host starten möchten, müssen wir Spring mitteilen, verschiedene Ports für den Tomcat Server zu verwenden.

Hierzu verwenden wir die \texttt{application.properties} Datei:

\begin{code}{properties}
    server.port = 8081
\end{code}


\subsubsection{Start}

Sobald wir die Main-Methode starten, wird der Spring Server initialisiert:

\begin{figure}
    \caption{Spring Console Output - Initialization}
    \includegraphics[width=15cm]{images/spring-console-init}
    \centering
\end{figure}

Außerdem wird hier die \texttt{/xml} \textbf{ReST Schnittstelle} "gemapped":

\begin{figure}
    \caption{Spring Console Output - XML ReST Interface Mapping}
    \includegraphics[width=15cm]{images/spring-console-mapped}
    \centering
\end{figure}

Zuletzt wird der Tomcat Server gestartet:

\begin{figure}
    \caption{Spring Console Output - Tomcat Server Start finish}
    \includegraphics[width=15cm]{images/spring-console-finish}
    \centering
\end{figure}

Die Spring Application ist nun erreichbar unter \texttt{localhost:8081/xml}.

\begin{figure}
    \caption{Spring Application Web-Aufruf (JSON)}
    \includegraphics[width=15cm]{images/spring-app-json}
    \centering
\end{figure}

\begin{code}{json}
    {"windpark": {
        "windrad": [
            {
                "transfertime": "850 ms",
                "bladeposition": "25.8 deg",
                "blindpower": "382.54 kWh",
                "temperature": "25.8 C",
                "windspeed": "40.54 km/h",
                "rotationspeed": "0.42 m/s",
                "id": "0001",
                "power": "752.51 kWh"
            },
            {
                "transfertime": "857 ms",
                "bladeposition": "23.1 deg",
                "blindpower": "281.54 kWh",
                "temperature": "23.1 C",
                "windspeed": "32.54 km/h",
                "rotationspeed": "0.25 m/s",
                "id": "0002",
                "power": "421.35 kWh"
            },
            {
                "transfertime": "777 ms",
                "bladeposition": "19.0 deg",
                "blindpower": "222.25 kWh",
                "temperature": "24.9 C",
                "windspeed": "29.10 km/h",
                "rotationspeed": "0.21 m/s",
                "id": "0003",
                "power": "251.2 kWh"
            }
        ],
        "id": "NOE0001"
    }}
\end{code}

Und falls es irgendeinen Fehler in der Applikation gibt:

\begin{figure}
    \caption{Spring Application Web-Aufruf - Error caught}
    \includegraphics[width=15cm]{images/spring-app-filenotfound}
    \centering
\end{figure}

Nun muss die Applikation zu einer \texttt{jar} gepackt werden, was mittels \textit{Artifacts} in \textit{IntelliJ IDEA} geschieht. Es wird also ein \texttt{jar} Artifact mit der Main-Class erstellt, welches dann das ausführen mehrerer Windparks unter verschiedenen Ports auf einem Rechner ermöglicht.


\clearpage
\subsection{Zentrale}

\subsubsection{Allgemein}

Nun erstellen wir eine \textbf{Spring Application} welche eine \textbf{Windpark Zentrale} repräsentiert.

Der Aufbau dieser \textbf{Spring Application} ist folgendermaßen gedacht:

\begin{itemize}
    \item Es werden mehrere Windparks eingetragen welche überwacht werden sollen. Diese Windparks werden mittels ihrer \textbf{ReST} Addresse eingetragen (also \textbf{IP}, \textbf{Port} und \textbf{Subseite} - beispielsweise: \texttt{172.0.1.2:8081/xml})
    \item Diese einzelnen Windparks werden über Ihre \textbf{ReST} Schnittstelle \textit{durch-iteriert}, und einzeln \textit{geparsed}.
    \item Nun werden die Windparks mittels Aufzeichnungsdatum fortlaufend auf einer \textbf{NoSQL MongoDB Datenbank} im \textbf{JSON} Format abgespeichert.
    \item Verwaltet werden diese Daten mittels den \textbf{MongoDB} Tools (Mongo Shell)
    \item Zusätzlich soll ein Webinterface erstellt werden, wo die Daten von der Mongo Datenbank angezeigt werden.
\end{itemize}

\subsubsection{Model}

Es wird ein \textbf{Model} definiert, welches die einzelnen Windräder, Windparks und Windpark Versionen als POJOs (\textbf{P}lain \textbf{O}ld \textbf{J}ava \textbf{O}bjects) repräsentiert.

Also werden die Klassen \texttt{Windpark.java}, \texttt{Windrad.java} und \texttt{WindparkVersion.java} erstellt, welche für den \textbf{MongoDB} Zugriff auch noch Mongo Repositories benötigen.

\begin{code}{java}
    public class Windpark {
        @Id
        private String id;

        private ArrayList<Windrad> windrad;

        public Windpark(String id) {
            this.id = id;
            this.windrad = new ArrayList<>();
        }

        public Windpark() {
        }

        public void addWindrad(Windrad windrad){
            this.windrad.add(windrad);
        }

        public ArrayList<Windrad> getWindrad() {
            return windrad;
        }

        public String getId() {
            return id;
        }

        @Override
        public boolean equals(Object o) {
            return o instanceof Windpark && ((Windpark) o).id.equals(this.id);
        }
    }
\end{code}

Ein Mongo Repository verwaltet die Schnittstelle zwischen der Java Runtime und dem MongoDB Treiber. Für den Java Code ist es also quasi die Schnittstelle zur Datenbank.

Eine Repository könnte beispielsweise folgendermaßen aussehen:

\begin{code}{java}
    public interface WindparkRepository extends MongoRepository<Windpark, String> {
        public Windpark findByid(String id);
    }
\end{code}

Die Spring Application kennt jedoch noch keine Mongo Datenbank, deshalb muss ein Connection String spezifiziert werden.

Hierzu verwenden wir die \texttt{application.properties} Datei:

\begin{code}{properties}
    spring.data.mongodb.host=172.17.0.2
    spring.data.mongodb.port=27017
    spring.data.mongodb.database=windpark
\end{code}

..dies ist der Mongo Docker Container welcher folgendermaßen gestartet werden kann:

\begin{code}{sh}
    docker run -p 27017:27017 --name mongodb -d mongo:latest
\end{code}



\clearpage
\subsubsection{Data Access Layer}

Um die einzelnen Daten nun von den Windparks zu bekommen, erstellen wir einen \textbf{Data Access Layer}, welcher folgende Aufgaben hat:

\begin{itemize}
    \item Windpark Zentrale Config lesen (zu überwachende Windparks auslesen)
    \item Windparks JSON auslesen über deren ReST Schnittstelle
    \item Einzelne Windparks Zusammenfassen
    \item Auf MongoDB Datenbank fortlaufend speichern
\end{itemize}

Es wird die Java Klasse \texttt{DataAccessLayer.java} erstellt, welche die einzelnen Mongo Repositories als Klassenvariablen verwendet:

\begin{code}{java}
    public class DataAccessLayer {
        // MongoDB Repositories
        private WindparkRepository windparks;
        private WindparkVersionRepository windparkVersions;
        private WindradRepository windrads;


        public DataAccessLayer(WindparkRepository windparks, WindparkVersionRepository windparkVersions, WindradRepository windrads) {
            this.windparks = windparks;
            this.windparkVersions = windparkVersions;
            this.windrads = windrads;
        }

        ...
\end{code}

..wobei der DataAccessLayer auch \texttt{@AutoWired} Repositories haben kann, also Repositories die von Spring automatisch mit einer Implementierung initialisiert werden.

Außerdem verwenden wir eine Konfigurationsdatei welche alle Windpark URLs enthält, diese wird folgendermaßen ausgelesen:

\begin{code}{java}
    private static final String FILE_NAME = "urls.txt";

    public ArrayList<String> getUrls() {
        try (BufferedReader br = new BufferedReader(new FileReader(FILE_NAME))) {
            ArrayList<String> urls = new ArrayList<>();

            String currentLine;
            while ((currentLine = br.readLine()) != null) {
                urls.add(currentLine.trim());
            }

            return urls;
        } catch (Exception e) {
            e.printStackTrace();
            return new ArrayList<>();
        }
    }
\end{code}

Die JSON Informationen der einzelnen Windparks können wir mittels folgender Funktion ermitteln:

\begin{code}{java}
    public static String getWindparkData(String url) throws URISyntaxException {
        RestTemplate rest = new RestTemplate();
        return rest.exchange(new URI(url), HttpMethod.GET, null, String.class).getBody();
    }
\end{code}

Hierbei ist aufzupassen, dass die URLs im richtigen format sind.

Es wird also stets eine \texttt{urls.txt} Datei im \textit{Working Directory} benötigt, welche beispielsweise folgendermaßen aufgebaut ist:

\begin{code}{properties}
    http://localhost:8081/json
    http://localhost:8082/json
    http://localhost:8083/json
    http://86.23.10.40/getWindparkInfo
\end{code}

Nun werden die einzelnen Windpark-Informationen von den Servern geholt:

\begin{figure}
    \caption{Spring App 5s Fetching}
    \includegraphics[width=15cm]{images/spring-app-fetching}
    \centering
\end{figure}









\clearpage
\subsubsection{ReST Schnittstelle}

Um einen Überblick der Zentrale zu bekommen, wird zusätzlich eine ReST Schnittstelle entworfen welche alle Windparks anzeigen kann.


\begin{code}{java}
    @RestController
    public class ZentraleRestController {
        private final WindparkRepository windparks;

        private final WindparkVersionRepository windparkVersions;

        private final WindradRepository windrads;

        @Autowired
        public ZentraleRestController(WindparkRepository windparks, WindparkVersionRepository windparkVersions, WindradRepository windrads) {
            this.windparks = windparks;
            this.windparkVersions = windparkVersions;
            this.windrads = windrads;
        }

        ...
\end{code}

Hierbei werden folgende ReST Schnittstellen implementiert:

\begin{code}{java}
    @GetMapping("/uniquewindparks")
    public List<Windpark> uniqueWindparks(){
        List<WindparkVersion> allWindparks = windparkVersions.findAll();
        List<Windpark> windparks = new ArrayList<>();
        for(WindparkVersion version: allWindparks){
            if(!windparks.contains(version.getWindpark())){
                windparks.add(version.getWindpark());
            }
        }
        return windparks;
    }

    @GetMapping("/windparks")
    public List<Windpark> windparks(){
        return windparks.findAll();
    }

    @GetMapping("/windpark")
    public Windpark windpark(@RequestParam(value="id") String id){
        return windparks.findByid(id);
    }

    @GetMapping("/fetch")
    public void fetch() {
        DataAccessLayer dal = new DataAccessLayer(windparks, windparkVersions, windrads);
        dal.fetch();
    }
\end{code}

Unter \texttt{localhost/windparks}:

\begin{figure}
    \caption{Spring App Windparks}
    \includegraphics[width=15cm]{images/spring-app-windparks}
    \centering
\end{figure}

Unter \texttt{localhost/windrads}:

\begin{figure}
    \caption{Spring App Windrads}
    \includegraphics[width=15cm]{images/spring-app-windrads}
    \centering
\end{figure}




\clearpage
\subsubsection{GUI}

Zusätzlich wird eine kleine grafische Oberfläche implementiert, welche in einer Listenform die Windparks anzeigt.

Hierzu erstellen wir die Websites \texttt{windpark.html} und \texttt{windparks.html} im \texttt{src/main/resources/templates} Ordner.

Es wird mit Thymeleaf, ein \textit{HTML Preprocessor} für Java, gearbeitet.

Hierbei wird für Variablen ein einfacher \textit{Placeholder} eingesetzt, welcher im Format von \texttt{\$\{variablenname\}} geschrieben wird.

\begin{code}{html}
    <div class="jumbotron">
        <h1 class="text-center">Alle Windparks</h1>
        <div class="row">
            <div class="col-sm-4" th:each="windpark : ${windparks}">
                <h1 class="text-center">Windpark <span th:text="${windpark.id}"></span></h1>
                <div class="row">
                    <div class="col-sm-4" th:each="windrad,iter : ${windpark.windrad}">
                        <div class="card" style="width: 38rem;">
                            <div class="card-body">
                                <div class="row">
                                    <div class="card" style="width: 17rem;">
                                        <div class="card-body" >
                                            <h5>Windrad: <span th:text="${windrad.id}"></span></h5>
                                            Id: <span th:text="${windrad.id}">id</span><br/>
                                            power: <span th:text="${windrad.power}">id</span><br/>
                                            blindpower: <span th:text="${windrad.blindpower}">id</span><br/>
                                            windspeed: <span th:text="${windrad.windspeed}">id</span><br/>
                                            rotationspeed: <span th:text="${windrad.rotationspeed}">id</span><br/>
                                            temperature: <span th:text="${windrad.temperature}">id</span><br/>
                                            bladeposition: <span th:text="${windrad.bladeposition}">id</span><br/>
                                            transfertime: <span th:text="${windrad.transfertime}">id</span>
                                        </div>
                                    </div>
                                </div>
                            </div>
                        </div>
                    </div>
                </div>
        </div>
    </div>
\end{code}

Für das einsetzen der Variablen-Werte müssen wir einen \textit{Controller} für die \textit{View} implementieren:

\begin{code}{java}
    @Controller
    public class ViewController {

        private final WindparkRepository windparks;
        private final WindparkVersionRepository windparkVersions;

        @Autowired
        public ViewController(WindparkRepository windparks, WindparkVersionRepository windparkVersions) {
            this.windparks = windparks;
            this.windparkVersions = windparkVersions;
        }

        ...
\end{code}

Es werden nun zwei \textbf{GET} mappings definiert:

\begin{code}{java}
    @GetMapping("/graphic/windparks")
    public String windparks(Model model){
        List<WindparkVersion> allWindparks = windparkVersions.findAll();
        List<Windpark> windparks = new ArrayList<>();
        for(WindparkVersion version: allWindparks){
            if(!windparks.contains(version.getWindpark())){
                windparks.add(version.getWindpark());
            }
        }

        model.addAttribute("windparks", windparks);

        return "windparks";
    }

    @GetMapping("/graphic/windpark")
    public String windpark(Model model, @RequestParam(value="id") String id){
        model.addAttribute("windpark", windparks.findByid(id));
        return "windpark";
    }
\end{code}

Beim Aufrufen von \texttt{localhost/graphic/windparks} erscheint nun folgende grafische Oberfläche:

\begin{figure}
    \caption{Spring App Website}
    \includegraphics[width=15cm]{images/spring-app-gui}
    \centering
\end{figure}





\clearpage
\subsubsection{MongoDB}

Die Mitarbeiter können außerdem auch in der MongoDB Shell alle Windparks anzeigen lassen:

\begin{figure}
    \caption{MongoDB all Windrads}
    \includegraphics[width=15cm]{images/mongo-windrads}
    \centering
\end{figure}

Die Mongo Abfragen schauen folgendermaßen aus:

\begin{code}{python}
    show databases
    use windpark

    show tables

    db.windpark.find().pretty()
    db.windrad.find().pretty()
    db.windparkVersion.find().pretty()
\end{code}



\clearpage
\section{Fragen}

\begin{itemize}
    \item Nennen Sie 5 Vorteile eines NoSQL Repository im Gegensatz zu einem relationalen DBMS
        \subitem 1. Skalierbarkeit
        \subitem 2. Exakt anpassbar auf gegebene Situation
        \subitem 3. Sehr einfach zu verwenden und verwalten
        \subitem 4. Gut aufteilbar in Cluster/Verteilte Systeme
        \subitem 5. Können große Datenmengen verwalten
    \item Nennen Sie 4 Nachteile eines NoSQL Repository im Gegensatz zu einem relationalen DBMS
        \subitem 1. Neue Technologie (Bugs)
        \subitem 2. Keine universelle Sprache wie \textbf{SQL}
        \subitem 3. Nicht genug Support
        \subitem 4. Nicht genug Hilfe
    \item Welche Schwierigkeiten ergeben sich bei der Zusammenführung der Daten?
        \subitem \textit{Unique Constraints} dürfen nicht verletzt werden, außerdem muss man aufpassen dass es immer Listen sind und mehrere Elemente (Windparks) zugelassen werden können.
    \item Können die Daten der MongoDB von Mitarbeitern geändert werden?
        \subitem Ja, da die \texttt{MongoRepository} sich um Änderungen der Datenbank kümmert.
    \item Beschreiben Sie die wichtigsten Eigenschaften des Spring Frameworks?
        \subitem Es gibt Spring Boot welches Templates bereitstellt
        \subitem Es gibt die AutoWired Annotation welches automatisch eine Implementierung bereitstellt
        \subitem Spring ist ein Web Framework, somit sind ReST Schnittstellen sehr schnell implementiert
    \item Was versteht man unter dem Spring Boot Projekt?
        \subitem Ein \textit{Spring Boot Projekt} ist ein Projekt, welches mittels dem \textit{Spring Boot} Framework automatisch Module und Komponenten eingefügt bekommt wo es benötigt wird. Es sind also Templates.
    \item Nennen Sie jeweils 3 Argumente für und gegen den Einsatz von Spring bei der Entwicklung solcher Projekte
        \subitem + 1. Es ist sehr schnell die Entwicklungsumgebung aufzusetzen
        \subitem + 2. Es besteht bereits eine gute Codebase, man muss "\textit{das Rad nicht neu erfinden}".
        \subitem + 3. Es bietet guten Support an
        \subitem - 1. Es kann nur Tomcat verwendet werden
        \subitem - 2. Andere Frameworks wie NodeJS bieten eine einfachere Umgebung
        \subitem - 3. Der Projektumfang ist zu gering für Spring
\end{itemize}
